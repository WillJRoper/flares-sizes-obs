\section{Conclusions}
\label{sec:conclusion}

In this paper we have presented an analysis of galaxy sizes at $z\geq5$ in the \flares\ simulations across a wide array of environments. To do this we produced synthetic galaxy images using photometry in rest frame UV and visual bands derived using the line of sight attenuation method presented in \cite{Vijayan2020}. We presented an efficient method of image computation by utilising a KD-Tree of pixel coordinates and smoothing stellar particles over their SPH kernels. We employed this imaging method to produce synthetic galaxy images, from which the size of galaxies were measured using a non-parametric pixel based method to account for the clumpy nature of galaxies at high redshift. 

Using these measurements we probed both the intrinsic and observed size-luminosity relation in the rest frame far-UV (1500 \AA), finding:

\begin{itemize}
    \item The intrinsic size-luminosity relation is bi-modal, with one intrinsically compact and bright population and one intrinsically diffuse and dim population.
    \item These 2 populations result in a negative slope to the rest-frame far-UV intrinsic size-luminosity distribution.
    \item Including the effects of dust attenuation results in the perceived size of galaxies to increase, with the most intrinsically compact galaxies increase in size by as much as $\times50$. 
    \item The increase in size due to dust attenuation inverts the slope of the size-luminosity relation, resulting in a fair agreement between observations and in this work. However, the \flares\ sample lacks low luminosity compact galaxies which have been shown to steepen the size-luminosity relation in lensing studies. Conversely, the observational samples lack the diffuse and dim galaxies that are present in this work, these act to flatten the size-luminosity relation. The affects of these missing galaxies highlights the need for high resolution simulations in the future and observationally motivated measurement methods.
    \item Dust distributions in these compact galaxies are highly concentrated with half metal radii of $<1$ pkpc, heavily attenuating the intrinsically bright cores and increasing the observed half light radius. This may be observable as strong dust gradients. 
\end{itemize}

We performed size measurements for a range of rest frame UV and visual bands, finding an anti-correlation between the slope of the size-luminosity relation and wavelength. This anti-correlation becomes weaker with decreasing redshift as the intrinsic stellar distribution increases in size. This represents a falsifiable prediction which Webb will be able to probe at high resolution with NIRCam.

We then investigated the evolution of size with redshift in the far-UV, finding slopes for multiple sample definitions in the range $m=1.21-1.87$. These values are consistent with theoretical predictions modified by additional contributions to the evolution by feedback mechanisms. At low luminosity the evolution is consistent with an evolution at fixed mass ($m=1$) with additional evolution due to feedback, while high luminosity galaxies are consistent with a fixed circular velocity evolution ($m=1.5$), again with an additional contribution from feedback. 
With the exception of the low luminosity sample giving a good agreement, these results are in tension with observations. They do however broadly agree with the range found in the FIRE-2 simulations. The limited observational galaxy sample at extremely high redshifts could contribute to this tension. 
Limiting the galaxy sample to both a low ($5 \leq z \leq 10$) and high ($7 \leq z \leq 12$) redshift sample yielded little change in the results for the low redshift sample but resulted in significantly higher slopes for the high redshift sample. This implies a non-constant size evolution with faster evolution in the highest redshift bins. Further observations from future high redshifts surveys are needed to probe the differences highlighted here in addition to future simulations adding to the theory. 

% To summarise, in the rest frame far UV (1500 \AA) galaxies at high redshift are intrinsically compact with a negative size-luminosity relation. This relation reverses with the inclusion of dust, which heavily attenuates their dense intrinsically bright cores, increasing their apparent size drastically. Comparison between these results with the effects of dust show plausible agreement between \flares\ and observations. 

With the launch of Webb we will soon be able to probe these high redshift regimes with far greater fidelity and further strengthen our understanding of the earliest epochs of galaxy evolution. Webb will allow us to probe higher redshifts at high resolution with NIRCam. Not only will this further populate galaxy samples at $z>8$, it will also increase the completeness of the high redshift observational surveys at low luminosity.

Future work will include the next generation of \flares\ simulating a wider range of environments, probing more regions, and simulating a significant volume at high mass resolution. Including higher resolution simulations will enable comparison to the dim and compact galaxies found in lensing studies, while increasing the effective volume with more resimulated regions will allow \flares\ to reach a volume comparable to the largest upcoming observational surveys from \euclid.   

In addition to the next generation of \flares, the underlying physical processes governing the size evolution in the subgrid model will be probed. This will include stellar and AGN feedback, star formation conditions and chemical enrichment. The effects of simulation and observational structure detection methods will be investigated to quantify the effect of survey depth and the segmentation of substructures. In particular this will aim to probe the effects of structure detection methods on the diffuse galaxy population and the effect this has on the size-luminosity relation.
